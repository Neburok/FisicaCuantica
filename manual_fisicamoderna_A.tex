\documentclass[oneside,12pt]{book}
\usepackage[utf8]{inputenc}
\usepackage[spanish,activeacute]{babel}
\usepackage{amsmath}
\usepackage{times}
\usepackage{graphicx}
\usepackage{afterpage}
\usepackage{fancyhdr} 
\usepackage{titlesec}
\usepackage{multirow}
\usepackage{xcolor}
\usepackage{tikz}
\usepackage{cancel}
\usepackage{geometry}
\usepackage{titling}

% Configuración de geometría global para el documento
\geometry{
	paperwidth=21.59cm,
	paperheight=27.94cm,
	top=2cm,
	bottom=2cm,
	left=2cm,
	right=2cm,
	headheight=15pt,
	headsep=0.5cm,
	footskip=1cm,
	includeheadfoot
}

% Colores UTEQ aproximados
\definecolor{uteqGreen}{RGB}{122, 181, 29}
\definecolor{uteqDarkBlue}{RGB}{0, 32, 91}

% Configuración de párrafos
\parindent 0em
\parskip 2ex   

% Configuración de páginas y encabezados
\fancyhf{}
\fancyfoot[RO]{\thepage} 
\renewcommand{\chaptermark}[1]{\markboth{\textit{\thechapter. #1}}{}} 
\renewcommand{\sectionmark}[1]{\markright{\textit{\thesection. #1}}} 

% Formato de capítulos
\newcommand{\bigrule}{\titlerule[0.5mm]} 
\titleformat{\chapter}[display]
{\bfseries\Huge} 
{% 
	\titlerule 
	\filleft
	\Large\chaptertitlename\
	\Large\thechapter} 
{0mm} 
{\filleft} 
[\vspace{0.5mm} \bigrule] 
\titlespacing{\chapter}{0mm}{-75pt}{20pt}

% Interlineado
\linespread{1}

% Fuentes
\renewcommand{\rmdefault}{phv}
\renewcommand{\sfdefault}{phv}

\begin{document}
	% Aplicar estilo vacío para la portada (sin números de página)
	\pagestyle{empty}
	
	% Redefinir nombre de capítulo  
	\renewcommand\chaptername{UNIDAD}
	
	\frontmatter
	
	% Aplicar geometría específica SOLO para la portada con margen superior reducido
	\newgeometry{
		paperwidth=21.59cm,
		paperheight=27.94cm,
		top=0.5cm, % Reducido de 2cm a 0.5cm
		bottom=2cm,
		left=2cm,
		right=2cm
	}
	
	% Portada con estilo moderno y ajustada a una página
	\begin{titlepage}
		\begin{center}
			
			% Línea decorativa superior
			\begin{tikzpicture}
				\fill[uteqGreen] (0,0) rectangle (\textwidth,0.3cm);
			\end{tikzpicture}
			
			\vspace{0.6cm}
			
			% Logo y nombre de universidad
			\begin{minipage}{0.3\textwidth}
				\includegraphics[width=\textwidth]{Figuras/LogoUTEQ}
			\end{minipage}
			\hfill
			\begin{minipage}{0.65\textwidth}
				\raggedleft
				{\fontsize{18}{22}\selectfont\textcolor{uteqDarkBlue}{\textbf{UNIVERSIDAD TECNOLÓGICA}}} \\
				{\fontsize{18}{22}\selectfont\textcolor{uteqDarkBlue}{\textbf{DE QUERÉTARO}}} \\
				\vspace{0.1cm}
				{\fontsize{9}{11}\selectfont\textcolor{uteqGreen}{Voluntad. Conocimiento. Servicio}}
			\end{minipage}
			
			\vspace{2cm}
			
			% Línea decorativa central
			\begin{tikzpicture}
				\draw[uteqGreen, line width=1.5pt] (0,0) -- (\textwidth,0);
			\end{tikzpicture}
			
			\vspace{0.5cm}
			
			% Programa educativo
			{\Large\textcolor{uteqDarkBlue}{Programa Educativo:}}
			
			\vspace{0.4cm}
			{\Large\textit{Ingeniería en Nanotecnología}}
			
			\vspace{2cm}
			
			% Título del manual
			{\fontsize{24}{28}\selectfont\textbf{\textcolor{uteqGreen}{FÍSICA MODERNA}}}
			
			\vspace{0.8cm}
			{\Large Manual de Asignatura \the\year}
			
			\vspace{3.5cm}
			
			% Información del autor - alineada más al centro
			\begin{minipage}{0.8\textwidth}
				\begin{flushright}
					\textbf{Autor:} \\
					Velázquez Hernández Rubén \\
					\vspace{0.3cm}
					\textbf{Fecha de publicación:} Enero \the\year
				\end{flushright}
			\end{minipage}
			
			\vfill
			
			% Línea decorativa inferior
			\begin{tikzpicture}
				\fill[uteqGreen] (0,0) rectangle (\textwidth,0.3cm);
			\end{tikzpicture}
			
		\end{center}
	\end{titlepage}
	
	% Restaurar la geometría original para el resto del documento
	\restoregeometry
	
	% Restaurar estilo fancy para el resto del documento
	\pagestyle{fancy}
	
	% Tabla de contenidos
	\tableofcontents
	
	% Incluir capítulos
	\chapter{Introducción}
	\begin{tabular}{|p{6cm}|p{9cm}|}
	\hline 
	\textbf{1. Nombre de la asignatura }& Física Cuántica \\
	\hline 
	\textbf{2. Competencias }& Diseñar procesos de producción de materiales nanoestructurados en laboratorio y a nivel industrial, con base en la planeación, técnicas de síntesis e incorporación y normatividad aplicable, para su comercialización y contribuir a la innovación tecnológica.
	
 \\
       \hline 
       \textbf{3. Cuatrimestre} & Octavo \\
       \hline 
       \textbf{4. Horas Prácticas} & 24 \\
       \hline
        \textbf{5. Horas Teóricas} & 36 \\
        \hline
       \textbf{6. Horas Totales} & 60 \\
        \hline
        \textbf{7. Horas Totales por semana cuatrimestre} & 4 \\
        \hline
        \textbf{8. Objetivo de la Asignatura} & El alumno describirá el comportamiento de los materiales
        nanoestructurados con base en los conceptos, teorías y principios de física moderna para determinar sus
        características y propiedades. \\
        \hline
	\end{tabular}
	
\begin{tabular}{|p{9.5cm}|c|c|c|}
\hline
\multirow{2}{*}{\textbf{Unidades Temáticas}} & \multicolumn{3}{c}{\textbf{Horas}} \\


  & \textbf{Prácticas} & \textbf{Teóricas} & \textbf{Totales} \\
  \hline
I. Teoría de la Relatividad
 & 4 & 6 & 10\\
\hline
II. Modelo Nuclear del Átomo & 4 & 6 & 10\\
\hline
III. Dualidad Onda-Partícula & 4 & 6 & 10 \\
\hline
IV. Teoría Básica del Electromagnetismo & 4 & 8 & 12 \\
\hline
V. Solución de la Ecuación Schröndinger & 8 & 10 & 18 \\
\hline
 & 24 & 36 & 60 \\
\hline


\end{tabular}

	\mainmatter

	\include{fundamentoscuantica}
	\include{dualidad}
	\chapter{ Solución de la Ecuación de Schrödinger }
\paragraph{Objetivo: }
El alumno determinará el comportamiento cuántico y electrónico de los materiales nanoestructurados para su aplicación.

\paragraph{Resultado de aprendizaje: }
A partir de un caso de estudio elaborará un reporte que incluya:

-Descripción del caso de estudio.

-Solución la ecuación de Schröendinger, en el átomo Hidrogeno.

-Definición de los estados cuánticos que determinen la diferencia entre material cristalino y uno no cristalino de acuerdo a
la teoría de bandas.
 

\section{Pozo de potencial
}
\paragraph{Saber: }
\textit{Definir los conceptos de Pozo de potencial y Barreras de potencial. }

\section{Efecto tunel}
\paragraph{Saber: }
\textit{Explicar el comportamiento 	de una partícula en un pozo de potencial. Definir la zona prohibida para el electrón.}

\section{Potenciales periódicos}
\paragraph{Saber: }
\textit{Explicar la distribución de	cargas propuesto por Kroning-Penny de un cristal unidimensional. Explicar la diferencia entre cristal perfecto y real}

\section{Ecuación de onda.}
\paragraph{Saber: }
\textit{Explicar la solución de la ecuación de Schrödinger 	para el átomo de Hidrógeno. Reconocer los niveles de energía. Describir la paradoja del gato de Schrödinger.}

\section{Estructura de 	bandas}
\paragraph{Saber: }
\textit{Reconocer los sólidos cristalinos, no cristalinos y cuasi cristalinos de acuerdo a la teoría de bandas en modelos de amarre fuerte.}

\section{Definición microscópica de conductores, semiconductores y aislantes}
\paragraph{Saber: }
\textit{Reconocer los materiales conductores, semiconductores y aislante de acuerdo a la teoría de bandas.}
	\include{atomosestructura}
	\chapter{Introducción al Estado Sólido}
\paragraph{Objetivo:}
Lorem ipsum dolor sit amet, consectetur adipiscing elit. Integer euismod odio eget leo bibendum aliquam. Nulla id leo nunc. Etiam maximus magna eget tristique varius. Maecenas venenatis cursus enim in tristique. Aliquam in eleifend tortor. Duis vehicula luctus lacus vitae venenatis. Donec sit amet condimentum magna. Integer mollis bibendum enim, quis consectetur lectus malesuada sed. Nulla pharetra magna in urna luctus tristique.

\paragraph{Resultado de aprendizaje: }

\begin{itemize}
	\item Lorem ipsum dolor sit amet, consectetur adipiscing elit.
	\item Donec a turpis ut orci vestibulum vestibulum non consectetur eros.
	\item Aliquam rutrum velit at velit rutrum congue.
	\item Donec sed diam porttitor, blandit metus eget, tempus lorem.
\end{itemize}


\section{Tema 1}
\paragraph{Saber: }
\textit{
	Lorem ipsum dolor sit amet, consectetur adipiscing elit. Integer euismod odio eget leo bibendum aliquam. Nulla id leo nunc. Etiam maximus magna eget tristique varius. Maecenas venenatis cursus enim in tristique. Aliquam in eleifend tortor. Duis vehicula luctus lacus vitae venenatis. Donec sit amet condimentum magna. Integer mollis bibendum enim, quis consectetur lectus malesuada sed. Nulla pharetra magna in urna luctus tristique.
}

\section{Tema 2}
\paragraph{Saber: }
\textit{
	Lorem ipsum dolor sit amet, consectetur adipiscing elit. Integer euismod odio eget leo bibendum aliquam. Nulla id leo nunc. Etiam maximus magna eget tristique varius. Maecenas venenatis cursus enim in tristique. Aliquam in eleifend tortor. Duis vehicula luctus lacus vitae venenatis. Donec sit amet condimentum magna. Integer mollis bibendum enim, quis consectetur lectus malesuada sed. Nulla pharetra magna in urna luctus tristique.
}

\section{Tema 3}
\paragraph{Saber: }
\textit{
	Lorem ipsum dolor sit amet, consectetur adipiscing elit. Integer euismod odio eget leo bibendum aliquam. Nulla id leo nunc. Etiam maximus magna eget tristique varius. Maecenas venenatis cursus enim in tristique. Aliquam in eleifend tortor. Duis vehicula luctus lacus vitae venenatis. Donec sit amet condimentum magna. Integer mollis bibendum enim, quis consectetur lectus malesuada sed. Nulla pharetra magna in urna luctus tristique.
}	
\end{document}