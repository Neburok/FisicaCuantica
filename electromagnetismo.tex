\chapter{Teoría Básica del Electromagnetismo}
\paragraph{Objetivo:}
El alumno determinará el comportamiento electromagnético de los materiales nanoestructurados para su caracterización.


\paragraph{Resultado de aprendizaje: }
A partir de un caso de estudio elaborará un reporte que incluya:

- La diferencia entre un campo magnético y un campo eléctrico.

-Solución de problemas con las ecuaciones de Maxwell.

- Relacionar las propiedades magnéticas con las eléctricas.

-Identificar materiales dieléctricos, semiconductores y conductores.

\section{Campos eléctricos y magnéticos.}
\paragraph{Saber: }
\textit{Explicar las magnitudes electromagnéticas. Definir los campos  eléctricos y magnéticos, y su efecto en las	propiedades de los materiales nanoestructurados.
}

\section{Ecuaciones de 	Maxwell.}
\paragraph{Saber: }
\textit{Explicar las ecuaciones de Maxwell. Relacionar los campos y los desplazamientos de una onda electromagnética.}

\section{Ecuación de onda y polarización de la luz.}
\paragraph{Saber: }
\textit{Identificar la ecuación de onda y su relación con la polarización de la luz.}

\section{Ondas planas en conductores y dieléctricos
}
\paragraph{Saber: }
\textit{Describir el concepto de ecuación de onda. Explicar la aplicación de la mecánica cuántica a los materiales nanoestructurados. }
