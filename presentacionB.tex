\documentclass[aspectratio=169]{beamer} % Relación 16:9 para presentaciones modernas

% Paquetes necesarios
\usepackage[utf8]{inputenc}
\usepackage[spanish]{babel}
\usepackage{graphicx}
\usepackage{tikz}
\usepackage{xcolor}
\usepackage{hyperref}

% Definición de colores institucionales UTEQ
\definecolor{uteqDarkBlue}{RGB}{3, 30, 60}  % Color azul oscuro del encabezado
\definecolor{uteqLightBlue}{RGB}{64, 148, 209}  % Azul claro de las líneas
\definecolor{uteqGreen}{RGB}{122, 181, 29}  % Verde institucional

% Configuración de tema beamer personalizado
\usetheme{default}
\usecolortheme{default}
\useinnertheme{default}
\useoutertheme{default}

% Eliminar iconos de navegación
\setbeamertemplate{navigation symbols}{}

% Configurar el pie de página
\setbeamertemplate{footline}{
	\begin{tikzpicture}[remember picture, overlay]
		% Círculo del logo en esquina inferior izquierda
		\fill[uteqDarkBlue] (0.8,0.8) circle (0.5cm);
		\fill[white] (0.8,0.8) circle (0.4cm);
		\fill[uteqDarkBlue] (0.8,0.8) circle (0.3cm);
		
		% Línea horizontal
		\draw[uteqLightBlue, line width=1.5pt] (1.5,0.8) -- (\paperwidth-0.5,0.8);
		
		% Número de diapositiva
		\node[anchor=east, text=uteqDarkBlue, font=\small] at (\paperwidth-0.5,0.8) {\insertframenumber/\inserttotalframenumber};
	\end{tikzpicture}
}

% Configurar el encabezado
\setbeamertemplate{headline}{
	\begin{tikzpicture}[remember picture, overlay]
		% Franja superior de colores
		\fill[uteqDarkBlue] (0,0) rectangle (\paperwidth,0.3cm);
		\fill[uteqLightBlue] (0,0.3cm) rectangle (\paperwidth,0.5cm);
		\fill[uteqGreen] (0,0.5cm) rectangle (\paperwidth,0.7cm);
		
		% Fondo azul oscuro para el título
		\fill[uteqDarkBlue] (0,0.7cm) rectangle (\paperwidth,2cm);
		
		% Logo UTEQ
		\node[anchor=west, inner sep=0] at (0.5,1.35) {\includegraphics[height=1cm]{logo-uteq.png}};
		
		% Línea vertical separadora
		\draw[white, line width=1pt] (3.5,0.9) -- (3.5,1.8);
		
		% Texto UTEQ
		\node[anchor=west, text=white, font=\Large\bfseries] at (4,1.35) {UNIVERSIDAD TECNOLÓGICA DE QUERÉTARO};
	\end{tikzpicture}
}

% Configurar el título centrado
\setbeamertemplate{frametitle}{
	\vspace*1.1cm} % Espacio para evitar solapamiento con el encabezado
	\begin{beamercolorbox}[wd=\paperwidth,leftskip=0.5cm,rightskip=0.5cm,sep=0.2cm]{frametitle}
		\usebeamerfont{frametitle}\insertframetitle
	\end{beamercolorbox}
}

% Colores del tema
\setbeamercolor{frametitle}{fg=uteqDarkBlue}
\setbeamercolor{title}{fg=uteqDarkBlue}
\setbeamercolor{subtitle}{fg=uteqDarkBlue}
\setbeamercolor{author}{fg=uteqDarkBlue}
\setbeamercolor{date}{fg=uteqDarkBlue}
\setbeamercolor{institute}{fg=uteqDarkBlue}
\setbeamercolor{normal text}{fg=uteqDarkBlue}
\setbeamercolor{itemize item}{fg=uteqGreen}
\setbeamercolor{itemize subitem}{fg=uteqLightBlue}
\setbeamercolor{block title}{bg=uteqDarkBlue,fg=white}
\setbeamercolor{block body}{bg=uteqLightBlue!20}

% Marca de agua con logo UTEQ (opcional, con baja opacidad)
\setbeamertemplate{background}{
	\begin{tikzpicture}[remember picture, overlay]
		\node[opacity=0.5, anchor=center, inner sep=0] at (0.5\paperwidth,0.45\paperheight) {\includegraphics[width=0.5\paperwidth]{logo-uteq.png}};
	\end{tikzpicture}
}

% Información del título
\title{\textbf{FÍSICA MODERNA}}
\subtitle{Ingeniería en Nanotecnología}
\author{Dr. Rubén Velázquez Hernández}
\date{\today}
\institute{Universidad Tecnológica de Querétaro}

\begin{document}
	
	% Diapositiva de título corregida
	\begin{frame}
		% Mantener el encabezado pero crear un título personalizado
		\vspace*{2cm} % Ajustar este valor para posicionar correctamente
		\begin{center}
			{\Large\bfseries\textcolor{uteqDarkBlue}{\inserttitle}}\\[0.5cm]
			{\large\textcolor{uteqDarkBlue}{\insertsubtitle}}\\[1cm]
			{\normalsize\textcolor{uteqDarkBlue}{\insertauthor}}\\[0.3cm]
			{\small\textcolor{uteqDarkBlue}{\insertinstitute}}\\[0.3cm]
			{\small\textcolor{uteqDarkBlue}{\insertdate}}
		\end{center}
	\end{frame}
	
	% Versión alternativa de la diapositiva de título
	\begin{frame}{FÍSICA MODERNA}
		\vspace{0.5cm}
		\begin{center}
			{\large\textbf{Ingeniería en Nanotecnología}}\\[1.5cm]
			
			\begin{columns}
				\begin{column}{0.6\textwidth}
					{\normalsize\textbf{Profesor:}}\\
					Dr. Rubén Velázquez Hernández\\[0.5cm]
					
					{\normalsize\textbf{Institución:}}\\
					Universidad Tecnológica de Querétaro\\[0.5cm]
					
					{\normalsize\textbf{Periodo:}}\\
					Enero - Abril \the\year
				\end{column}
				
				\begin{column}{0.35\textwidth}
					\includegraphics[width=\textwidth]{logo-uteq.png}
				\end{column}
			\end{columns}
		\end{center}
	\end{frame}
	
	% Ejemplo de diapositiva con contenido
	\begin{frame}{FUNDAMENTOS DE FÍSICA CUÁNTICA}
		\begin{itemize}
			\item La física cuántica estudia el comportamiento de la materia a escala atómica y subatómica
			\item Principios fundamentales:
			\begin{itemize}
				\item Dualidad onda-partícula
				\item Principio de incertidumbre
				\item Superposición cuántica
			\end{itemize}
			\item Aplicaciones en nanotecnología:
			\begin{itemize}
				\item Efectos de confinamiento cuántico
				\item Propiedades emergentes a escala nano
				\item Fenómenos de tunelamiento
			\end{itemize}
		\end{itemize}
	\end{frame}
	
	% Ejemplo de diapositiva con bloques
	\begin{frame}{POSTULADOS DE LA MECÁNICA CUÁNTICA}
		\begin{block}{Postulado 1: Estados Cuánticos}
			Todo sistema físico está representado por un vector de estado $|\Psi\rangle$ en un espacio de Hilbert.
		\end{block}
		
		\begin{block}{Postulado 2: Observables}
			Toda magnitud física medible A está representada por un operador hermítico $\hat{A}$.
		\end{block}
		
		\begin{block}{Postulado 3: Resultado de Mediciones}
			Al medir un observable, el resultado será uno de los autovalores $a_i$ del operador $\hat{A}$.
		\end{block}
	\end{frame}
	
	% Ejemplo de diapositiva con ecuaciones
	\begin{frame}{ECUACIÓN DE SCHRÖDINGER}
		La ecuación de Schrödinger independiente del tiempo:
		
		\begin{equation}
			-\frac{\hbar^2}{2m}\nabla^2\Psi(\mathbf{r}) + V(\mathbf{r})\Psi(\mathbf{r}) = E\Psi(\mathbf{r})
		\end{equation}
		
		Para una partícula en una dimensión:
		
		\begin{equation}
			-\frac{\hbar^2}{2m}\frac{d^2\Psi(x)}{dx^2} + V(x)\Psi(x) = E\Psi(x)
		\end{equation}
		
		\begin{itemize}
			\item $\Psi$ es la función de onda
			\item $V$ es la energía potencial
			\item $E$ es la energía total
		\end{itemize}
	\end{frame}
	
	% Ejemplo de diapositiva con dos columnas
	\begin{frame}{COMPARACIÓN DE MODELOS ATÓMICOS}
		\begin{columns}
			\begin{column}{0.48\textwidth}
				\textbf{Modelo de Bohr}
				\begin{itemize}
					\item Órbitas circulares discretas
					\item Niveles de energía cuantizados
					\item No explica átomos multielectrónicos
				\end{itemize}
				\vspace{0.5cm}
				\centering
				$E_n = -\frac{13.6\text{ eV}}{n^2}$
			\end{column}
			
			\begin{column}{0.48\textwidth}
				\textbf{Modelo Cuántico}
				\begin{itemize}
					\item Orbitales - regiones de probabilidad
					\item Números cuánticos (n, l, m, s)
					\item Compatible con el principio de incertidumbre
				\end{itemize}
				\vspace{0.5cm}
				\centering
				$\Psi_{nlm}(r,\theta,\phi) = R_{nl}(r)Y_{lm}(\theta,\phi)$
			\end{column}
		\end{columns}
	\end{frame}
	
\end{document}