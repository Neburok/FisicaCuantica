\chapter{Teoría de la Relatividad}
\paragraph{Objetivo:}
El alumno empleará los conceptos de la teoría de la relatividad para comprender los fenómenos físicos relacionados con la teoría cuántica.

\paragraph{Resultado de aprendizaje:}
Elaborará un ensayo que incluya:
Comparativo entre la física clásica y moderna. Definiciones, electromagnetismo, simultaneidad. El diagrama representativo del experimento de Michel-Morley.
 

\section{Introducción a la teoría de la Relatividad}
\paragraph{Saber:}
\textit{Definir la Teoría de la
	relatividad de Einstein. Definir la Teoría de la relatividad especial. La
	relatividad de la simultaneidad. Describir el experimento de Michelson-Morley. El principio de
	relatividad de Galileo y Newton}

En 1905, Albert Einstein publicó tres artículos que revolucionaron la forma en como se entiende la naturaleza. Uno relacionado al movimiento browniano; un segundo que se refería al efecto fotoeléctrico y un tercero que donde planteo su \textbf{teoría especial de la relatividad}. La cual plantea dos postulados, que se describen basándose en marcos de referencia inerciales. 

El primer postulado de Einstein, conocido como el principio de relatividad, afirma que \textbf{las leyes de la física son las mismas en todos los marcos de referencia inerciales.}

 
\section{La transformación galileana y la
		teoría electromagnética.}
\paragraph{Saber:}
\textit{Explicar la	transformación 	galileana y la teoría electromagnética. Las transformaciones de Lorentz y el espacio tiempo.}

\section{Comparación entre Física Clásica y Física Moderna.}
\paragraph{Saber:}
\textit{Explicar las diferencias entre la Física Clásica y la Física Moderna. Describir los fenómenos no explicados desde el enfoque clásico, por medio del enfoque cuántico.}