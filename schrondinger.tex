\chapter{ Solución de la Ecuación de Schrödinger }
\paragraph{Objetivo: }
El alumno determinará el comportamiento cuántico y electrónico de los materiales nanoestructurados para su aplicación.

\paragraph{Resultado de aprendizaje: }
A partir de un caso de estudio elaborará un reporte que incluya:

-Descripción del caso de estudio.

-Solución la ecuación de Schröendinger, en el átomo Hidrogeno.

-Definición de los estados cuánticos que determinen la diferencia entre material cristalino y uno no cristalino de acuerdo a
la teoría de bandas.
 

\section{Pozo de potencial}
\paragraph{Saber: }
\textit{Definir los conceptos de Pozo de potencial y Barreras de potencial. }

\section{Efecto tunel}
\paragraph{Saber: }
\textit{Explicar el comportamiento 	de una partícula en un pozo de potencial. Definir la zona prohibida para el electrón.}

\section{Potenciales periódicos}
\paragraph{Saber: }
\textit{Explicar la distribución de	cargas propuesto por Kroning-Penny de un cristal unidimensional. Explicar la diferencia entre cristal perfecto y real}

\section{Ecuación de onda.}
\paragraph{Saber: }
\textit{Explicar la solución de la ecuación de Schrödinger 	para el átomo de Hidrógeno. Reconocer los niveles de energía. Describir la paradoja del gato de Schrödinger.}

\section{Estructura de 	bandas}
\paragraph{Saber: }
\textit{Reconocer los sólidos cristalinos, no cristalinos y cuasi cristalinos de acuerdo a la teoría de bandas en modelos de amarre fuerte.}

\section{Definición microscópica de conductores, semiconductores y aislantes}
\paragraph{Saber: }
\textit{Reconocer los materiales conductores, semiconductores y aislante de acuerdo a la teoría de bandas.}