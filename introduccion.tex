\chapter{Introducción}

\begin{tabular}{|p{6cm}|p{10cm}|}
	\hline 
	\textbf{1. Nombre de la asignatura }& Física Moderna \\
	\hline 
	\textbf{2. Competencias }& Diseñar procesos de producción de materiales nano-estructurados en laboratorio y a nivel industrial, con base en la planeación, técnicas de síntesis e incorporación y normatividad aplicable, para su comercialización y contribuir a la innovación tecnológica.
	
 \\
       \hline 
       \textbf{3. Cuatrimestre} & Noveno \\
       \hline 
       \textbf{4. Horas Prácticas} & 24 \\
       \hline
        \textbf{5. Horas Teóricas} & 36 \\
        \hline
       \textbf{6. Horas Totales} & 60 \\
        \hline
        \textbf{7. Horas Totales por semana cuatrimestre} & 4 \\
        \hline
        \textbf{8. Objetivo de la Asignatura} & El alumno describirá el comportamiento de los materiales nanoestructurados con base en los conceptos, teorías y principios de física moderna para determinar sus características y propiedades\\
        \hline
        \textbf{8a. Objetivo de la Asignatura (modificado)} & El alumno describirá el comportamiento de los materiales nanoestructurados con base en los conceptos, teorías y principios de física moderna para determinar sus características y propiedades, utilizando herramientas computacionales y de inteligencia artificial como apoyo.\\
        \hline
	\end{tabular}
	
\begin{tabular}{|p{9.5cm}|c|c|c|}
\hline
\multirow{2}{*}{\textbf{Unidades Temáticas (oficial)}} & \multicolumn{3}{c}{\textbf{Horas}} \\


  & \textbf{Prácticas} & \textbf{Teóricas} & \textbf{Totales} \\
  \hline
I. Teoría Básica del Electromagnetismo 
 & 6 & 8 & 14\\
\hline
II. Modelo Nuclear del Átomo & 6 & 10 & 16\\
\hline
III. Dualidad Onda-Partícula & 4 & 6 & 10 \\
\hline
IV. Solución de la Ecuación Schröndinger & 8 & 12 & 20 \\
\hline
 & 24 & 36 & 60 \\
\hline

\end{tabular}

\begin{tabular}{|p{9.5cm}|c|c|c|}
	\hline
	\multirow{2}{*}{\textbf{Unidades Temáticas (propuesto)}} & \multicolumn{3}{c}{\textbf{Horas}} \\
	
	
	& \textbf{Prácticas} & \textbf{Teóricas} & \textbf{Totales} \\
	\hline
	I. Fundamentos de la Teoría Cuántica	& 5 & 7 & 12\\
	\hline
	II. Dualidad Onda-Partícula & 4 & 6 & 10 \\
	\hline
	III. Solución de la Ecuación Schröndinger & 6 & 8 & 14 \\
	\hline
	IV. Átomos y Estructura  & 4 & 6 & 10 \\
	\hline
	V. Introducción Estado Sólido & 6 & 8 & 14 \\
	\hline
	& 25 & 35 & 60 \\
	\hline
	
\end{tabular}