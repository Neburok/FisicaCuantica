\chapter{Introducción}
	\begin{tabular}{|p{6cm}|p{9cm}|}
	\hline 
	\textbf{1. Nombre de la asignatura }& Física Cuántica \\
	\hline 
	\textbf{2. Competencias }& Diseñar procesos de producción de materiales nanoestructurados en laboratorio y a nivel industrial, con base en la planeación, técnicas de síntesis e incorporación y normatividad aplicable, para su comercialización y contribuir a la innovación tecnológica.
	
 \\
       \hline 
       \textbf{3. Cuatrimestre} & Octavo \\
       \hline 
       \textbf{4. Horas Prácticas} & 24 \\
       \hline
        \textbf{5. Horas Teóricas} & 36 \\
        \hline
       \textbf{6. Horas Totales} & 60 \\
        \hline
        \textbf{7. Horas Totales por semana cuatrimestre} & 4 \\
        \hline
        \textbf{8. Objetivo de la Asignatura} & El alumno describirá el comportamiento de los materiales
        nanoestructurados con base en los conceptos, teorías y principios de física moderna para determinar sus
        características y propiedades. \\
        \hline
	\end{tabular}
	
\begin{tabular}{|p{9.5cm}|c|c|c|}
\hline
\multirow{2}{*}{\textbf{Unidades Temáticas}} & \multicolumn{3}{c}{\textbf{Horas}} \\


  & \textbf{Prácticas} & \textbf{Teóricas} & \textbf{Totales} \\
  \hline
I. Teoría de la Relatividad
 & 4 & 6 & 10\\
\hline
II. Modelo Nuclear del Átomo & 4 & 6 & 10\\
\hline
III. Dualidad Onda-Partícula & 4 & 6 & 10 \\
\hline
IV. Teoría Básica del Electromagnetismo & 4 & 8 & 12 \\
\hline
V. Solución de la Ecuación Schröndinger & 8 & 10 & 18 \\
\hline
 & 24 & 36 & 60 \\
\hline


\end{tabular}