\chapter{Dualidad Onda-Partícula}
\paragraph{Objetivo:}
El alumno interpretará el comportamiento dual onda-partícula de los fenómenos para caracterizar materiales nanoestructurados

\paragraph{Resultado de aprendizaje:}
A partir de un caso de estudio elaborará un reporte integrado de por lo menos tres prácticas que contenga:

-La diferencia entre un electrón, fotón y fonón.

- Resolver ejemplos de la dualidad onda-partícula.

-Análisis de la interacción Radiación-Materia.

-Cálculos de la energía emitida por un cuerpo aplicando la ley de cuerpo negro de Plank-Boltzman.

\section{Efecto Fotoeléctrico}

\paragraph{Saber: }
\textit{Describir la diferencia entre electrones, fonones y fotones. Describir el efecto	fotoeléctrico.}

\section{Hipótesis de De Broglie}

\paragraph{Saber: }
\textit{Describir el comportamiento de dual de la materia onda-partícula.}

\section{Interacción Radiación-Materia.}
\paragraph{Saber: }
\textit{Identificar los elementos de un espectro de Rayos-X. Describir la interacción Radiación-Materia. Reconocer la Interacción de partículas cargadas con la materia a través de colisiones elásticas e inelásticas con los núcleos atómicos. }

\section{Postulado de Planck y Radiación de
	cuerpo negro}
\paragraph{Saber: }
\textit{Reconocer el postulado de Plank y la ley de Steffan-Boltzman de radiación de cuerpo negro.}

